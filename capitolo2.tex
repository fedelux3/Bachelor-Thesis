\chapter{Stato dell'Arte}
\label{Teoria}
\thispagestyle{empty}

\section{Paper}
Come precedentemente anticipato, l'algoritmo oggetto di questo lavoro \`e descritto in: 
\begin{quotation}
	\textbf{A Novel Breadth-first Strategy Algorithm for\\ Discovering Sequential Patterns from Spatio-temporal Data} (Piotr S. Maciag e Robert Bembenik, 2019).
\end{quotation} 
Di seguito vi \`e la base teorica su cui si fonda l'algoritmo.

\section{Vicinato}
Il problema che si considera \`e la scoperta di pattern da un certo dataset di istanze di eventi, i quali sono di una data tipologia.\\
Definiamo quindi:

$D \rightarrow$ dataset di istanze di eventi

$F \rightarrow$ insieme di tipologie di eventi
\\\\
Ogni istanza $e \in D$ ha:
\begin{itemize}
	\item chiave di identificazione (unica)
	\item location spaziale (es. coordinate geografiche)
	\item istante temporale
	\item tipologia $f \in F$
\end{itemize}
\noindent
La sequenza di eventi (pattern) \`e cos\`i definita:

$\overrightarrow{s} = f_{i_1} \rightarrow f_{i_2} \rightarrow ... \rightarrow f_{i_n}$, dove $f_{i_1}, f_{i_2}, ..., f_{i_n} \in F$\\
Quindi per ogni due tipi consecutivi di eventi in una sequenza $f_{i_{j-1}} \rightarrow f_{i_j}$, le istanze dell'evento di tipo $j-1$ sono connesse con le istanze del tipo successivo $j$ spazialmente e temporalmente. \\
L'insieme di eventi collegati in questo modo a una determinata istanza viene definito \textbf{neighborhood} o  vicinato. 

\underline{Esempio}\\
Consideriamo una situazione come quella in Fig. 2.1, dove:\\
$D = \{a1,a2,b1,b2,b3,b4,b5,b6,b7,b8,c1,c2,c3,c4\}$\\
$F = \{A,B,C\}$
\begin{figure}[h!]
	\centering
	\includegraphics[height=0.6 \linewidth]{neighborhood.png}
	\caption{esempio di istanze con vicinato degli eventi di tipo A}
\end{figure}
\\
Una sequenza significativa per esempio potrebbe essere $\overrightarrow{s} = A \rightarrow B \rightarrow C$.\\
Per valutare la connessione $A \rightarrow B$ bisogna considerare il \textit{neighborhood} tra le loro istanze. Come si nota dalla figura per la dimensione del vicinato \`e stato scelto un raggio spaziale pari a $R = 3.5$ e un intervallo temporale pari a $T = 10$. 

\section{Nozioni di base}
Dopo aver inquadrato graficamente il problema, ora definiamo formalmente i concetti base usati per il calcolo di tutte le sequenze pattern impiegati nell'algoritmo.

\subsection{Spazio Neighborhood} Con $V_{N(e)}$ denotiamo lo \textit{spazio di neighborhood} (vicinato) dell'istanza $e$. Questo spazio si basa su tre dimensioni, che sono le due dimensioni spaziali - latitudine e longitudine - e la dimensione temporale. Graficamente ne risulta un cilindro, con i seguenti parametri:
\begin{itemize}
	\item $R$: il raggio spaziale 
	\item $T$: l'intervallo temporale
\end{itemize} 

In Figura 2.2 vengono mostrati i due neighborhood tratti dall'\textit{esempio} della Figura 2.1: $V_{N(a1)}$ e $V_{N(a2)}$.

\begin{figure}[h!]
	\centering
	\includegraphics[height=0.2 \linewidth]{neia1.png}\hfil
	\includegraphics[height=0.2 \linewidth]{neia2.png}
	\caption{$V_{N(a1)}$ e $V_{N(a2)}$}
\end{figure}
\clearpage
\subsection{Neighborhood rispetto ad una tipologia di \\ evento} Data una certa istanza $e$, il \textit{neighborhood} di $e$ rispetto ad istanze di tipo $f$ \`e definito nel modo seguente:\\

$N_f(e) = \{e|p \in D(f) $

\qquad \qquad $\wedge$ $distance(p.location, e.location) \leq R$ 

\qquad \qquad $\wedge$ $(p.time - e.time) \in [0,T]\}$\\
\\
dove $R$ e $T$ sono i parametri dello spazio di vicinato $V_{N(e)}$ e $D(f)$ \`e l'insieme di istanze degli eventi di tipo $f$ nel dataset $D$.\\
\\
\underline{Nota} si considerano solo gli eventi che si susseguono dal punto di vista temporale, in quanto \`e poco significativo nella ricerca di sequenze considerare gli eventi passati dell'istanza.
\\
\\
Nel nostro \textit{esempio} della Figura 2.2, cerco gli eventi di tipo $B$ presenti in $V_{N(a1)}$ e $V_{N(a2)}$: 

$N_B(a1) = \{b1,b2,b3\}$

$N_B(a2) = \{b4,b5,b6\}$

\subsection{Set di istanze} Per una sequenza di tipi di eventi $\overrightarrow{s} = \overrightarrow{s}[1] \rightarrow \overrightarrow{s}[2] \rightarrow ... \rightarrow \overrightarrow{s}[m]$ di lunghezza $m$, gli \textbf{insiemi (set) di istanze} $I(\overrightarrow{s}[1]),I(\overrightarrow{s}[2]),...,I(\overrightarrow{s}[m])$ che sono inclusi nella sequenza $\overrightarrow{s}$ sono definiti come segue: 
\begin{enumerate}
	\item Per un tipo di evento $\overrightarrow{s}[1]$, il set di istanze $I(\overrightarrow{s}[1])$ \`e definito come:

	\qquad $I(\overrightarrow{s}[1]) = D(\overrightarrow{s}[1])$
	
	\item Per i tipi $\overrightarrow{s}[2] \rightarrow ... \rightarrow \overrightarrow{s}[m]$ con $i = 2,3, ..., m$, gli insiemi di istanze $I(\overrightarrow{s}[i]$ sono definiti cos\`i:
	
	\qquad $I(\overrightarrow{s}[i]) = distinct( \underset{e \in I(\overrightarrow{s}[i-1])}{\bigcup} N_{\overrightarrow{s}[i]}(e))$	
\end{enumerate}
\noindent
In altre parole, per il primo tipo di evento (d'ora in poi nominato solo "tipo") che partecipa alla sequenza $\overrightarrow{s}$, il set di istanze $I(\overrightarrow{s}[1])$ corrisponde al set di istanze di tipo $\overrightarrow{s}[1]$ in $D$, ovvero $D(\overrightarrow{s}[1])$.\\
Per i tipi successivi di $\overrightarrow{s}$, i set $I(\overrightarrow{s}[i])$ sono definiti come insiemi di istanze contenute nei vicinati di istanze a partire da $I(\overrightarrow{s}[i-1])$.\\
Seguendo questo meccanismo si valuta tutta la sequenza, tenendo in considerazione l'insieme di istanze calcolato al passaggio precedente.
\\\\
\textit{Esempio}\\
Consideriamo la sequenza $\overrightarrow{s} = A \rightarrow B$ dal dataset dell'\textit{esempio} in Figura 2.1. In questo caso avremmo i seguenti set di istanze:\\
$I(\overrightarrow{s}[1]) = \{a1,a2\}$\\
$I(\overrightarrow{s}[2]) = \{b1,b2,b3,b4,b5,b6\}$

\subsection{Participation Ratio} Data una sequenza $\overrightarrow{s} = \overrightarrow{s}[1] \rightarrow \overrightarrow{s}[2] \rightarrow ... \rightarrow \overrightarrow{s}[m]$ il \textbf{participation ratio} tra due tipi consecutivi contenuti in $\overrightarrow{s}$ \`e definito:\\

$PR(\overrightarrow{s}[i-1] \rightarrow \overrightarrow{s}[i]) =$ {\large$\frac{|I(\overrightarrow{s}[i])|}{|D(\overrightarrow{s}[i])|}$}\\
\\
il valore corrisponde al numero di istanze distinte di tipo $\overrightarrow{s}[i]$ contenute nei neighborhoods delle istanze di tipo $\overrightarrow{s}[i-1]$ diviso il numero di istanze di tipo $\overrightarrow{s}[i]$ presenti nel dataset $D$.\\
Per ogni coppia di tipi consecutivi ($\overrightarrow{s}[i-1]$, $\overrightarrow{s}[i]$) in una sequenza $\overrightarrow{s}$ il \textit{participation rateo} \`e definito come il rapporto tra $|I(\overrightarrow{s}[i])|$ e $|D(\overrightarrow{s}[i])|$ e il suo valore \`e compreso nel range $[0,1]$.
\clearpage
\subsection{Participation Index} Data una sequenza lunga $m$: $\overrightarrow{s} = \overrightarrow{s}[1] \rightarrow \overrightarrow{s}[2] \rightarrow ... \rightarrow \overrightarrow{s}[m]$, il \textit{participation index} \`e cos\`i definito:
\begin{enumerate}
	\item se $m = 2$:
	
	\qquad $PI(\overrightarrow{s}) = PR(\overrightarrow{s}[1] \rightarrow \overrightarrow{s}[2])$
	
	\item se $m > 2$:
	
	\qquad $PI(\overrightarrow{s}) = min
	\begin{cases}
		PI(\overrightarrow{s}^\ast) \\ PR(\overrightarrow{s}[m-1] \rightarrow \overrightarrow{s}[m])
	\end{cases} 
	$
\end{enumerate}
dove $\overrightarrow{s}^\ast = \overrightarrow{s}[1] \rightarrow \overrightarrow{s}[2] \rightarrow ... \rightarrow \overrightarrow{s}[m-1]$\\
\\
Il \textbf{participation index} corrisponde al minimo di tutti i \textit{participation ratio} calcolati su tutte le coppie di tipi consecutivi presenti in $\overrightarrow{s}$, ed \`e il nostro valore di output dell'algoritmo.\\
Intuitivamente lo possiamo pensare come il numero di istanze di eventi collegate alle istanze di tipi connessi precedentemente secondo un ordine stabilito dalla sequenza e questo ci d\`a la misura di quanto la sequenza sia correlata.\\
\\
Consideriamo il nostro \textit{esempio} della Figura 2.1, per la sequenza $\overrightarrow{s} = A \rightarrow B$ il $PI(\overrightarrow{s}) = 0.75$, infatti $PI(A \rightarrow B) = PR(A \rightarrow B) = \frac{6}{8} = 0.75$, che \`e un buon risultato di correlazione. 

