\chapter{Base teorica}
\label{Teoria}
\thispagestyle{empty}

\section{Paper}
Come precedentemente anticipato l'algoritmo oggetto di questo lavoro \`e \\
\textbf{A Novel Breadth-first Strategy Algorithm for Discovering Sequential Patterns from Spatio-temporal Data} \\ di Piotr S. Maciag e Robert Bembenik del \textit{Instituite of Computer Science, Warsaw University of Tecnology, Nowowiejska 15/19, 00-665, Warsaw, Poland}.\\ Di seguito vi \`e la trascritta la base teorica su cui si fonda l'algoritmo e le strutture dati utilizzate per la sua realizzazione.

\section{Nozioni base}
Il problema che si considera \`e quello della scoperta di pattern da un certo dataset di istanze di eventi, i quali sono di una certa tipologia, definiamo quindi:

$D \rightarrow$ dataset di istanze di eventi

$F \rightarrow$ insieme di tipologie di eventi
\\\\
Ogni istanza $e \in D$ ha:
\begin{itemize}
	\item chiave di identificazione (unica)
	\item location spaziale (es. coordinate geografiche)
	\item istante temporale
	\item tipologia $f \in F$
\end{itemize}
\noindent
La sequenza di eventi (pattern) \`e cos\`i definita:

$\vec{s} = f_{i_1} \rightarrow f_{i_2} \rightarrow ... \rightarrow f_{i_n}$, dove $f_{i_1}, f_{i_2}, ..., f_{i_n} \in F$\\
Quindi per ogni due tipologie di eventi consecutive in una sequenza $f_{i_{j-1}} \rightarrow f_{i_j}$, le istanze dell'evento $j-1$ sono connesse con con il tipo successivo spazialmente e temporalmente, definendo una sorta di \textit{vicinato} comune tra istanze. 
\\\\
QUA FAI L'ESEMPIO DEL PAPER PER SPiEGARE COSA SI INTENDE (QUESTO \`E IL CAPITOLO DEL VICINATO???)
\section{Vicinato}

\section{Parametri di threshold}

\section{Struttura ad albero}

\section{Alternativa - Algoritmo apriori}

