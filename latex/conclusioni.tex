\chapter*{Conclusioni e prospettive future}
\label{Conclusioni}
\pagestyle{fancy}
\fancyhf{}
\rhead{\textit{Conclusioni e prospettive future}}
\cfoot{\thepage}

\addcontentsline{toc}{chapter}{Conclusioni e prospettive future}

Il lavoro svolto mostra come l'analisi di associazioni di eventi pu\`o portare ad importanti risultati anche nel campo dello studio e lotta al crimine. Il software realizzato \`e comunque grezzo e si apre a diverse migliorie, alcune idee potrebbero essere:
\begin{itemize}
	\item implementare un'interfaccia grafica che permetta di effettuare i test in modo interattivo, anche per quanto riguarda l'utilizzo di dataset diversi	
	\item implementare una mappa in cui vengono mostrati gli eventi connessi seguendo una determinata sequenza trovata, in modo da visualizzare graficamente le correlazioni
	\item migliorare la versatilit\`a del software e dare la possibilit\`a di scegliere diversi metodi di calcolo della distanza (es. strada fisica da percorrere), nonch\`e tagli temporali personalizzati
	\item adattarlo a concentrarsi solo su determinati quartieri della citt\`a, magari quelli ad alto tasso di criminalit\`a, in modo da tentare di far emergere pattern di criminalit\`a organizzata locale (gang)
\end{itemize}

\paragraph{Altre prospettive} L'algoritmo oggetto di questo elaborato lo \textit{Spatio-temporal Breadth First Miner} si apre a svariate possibilit\`a di applicazioni, non solo l'analisi di eventi criminali.\\ 
Si pu\`o scegliere il dataset da utilizzare, i cui eventi devono avere le quattro caratteristiche: una chiave di identificazione unica, la location spaziale, l'istante temporale e la tipologia di evento.\\
\\\\
Un esempio di applicazione pu\`o essere lo studio di \underline{eventi epidemiologici}, in cui ho un database di soggetti che hanno contratto una certa patologia con luogo e tempo della scoperta (naturalmente mantenendo l'anonimato). In questo caso si pu\`o adattare l'algoritmo per comprendere se certe patologie siano legate ad altre ed in quale misura.\\
